\documentclass [12pt]{beamer}
\usepackage{xcolor}	
\usepackage{tikz}
\usetheme{Warsaw}
\useoutertheme{infolines}
\usepackage{ragged2e}
\begin{document}
\section*{kholase safahat 4...6}
\subsection*{seyede fatemeh rostamiyan}	
\begin{frame}
\justifying	
The convergence of audio, video, and multimedia channels to a Net-based platform, which is still shrinking and increasing in power, has led to the exploitation of applications in almost every field.

In general, the prefix e means that the changed activity or name is performed on a high-speed digital network that is available "anytime / anywhere".  Today it is the Internet.
\end{frame}	
\begin{frame}
\justifying	
 We expect that the increasing power and ubiquity of the network, along with its imaginary use by researchers, will lead to the expansion and change in the scope of research conducted around the world.  The special task of electronic research The world of networks is full of large amounts of data.

Because the Net properly supports synchronous communication, it is not surprising that e-research
\end{frame}
\begin{frame}
\justifying	
uses this capability to provide a wide range of research methods and tool capacity.  Research applications can be used to use concurrent or asynchronous formats - or both.

Electronic rescue agents are able to use research tools, monitor activity and collect data without having to travel long distances or coordinate local schedules.

\end{frame}
\end{document}		