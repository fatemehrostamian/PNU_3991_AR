
\documentclass [10pt,a4paper]{book}

\begin{document}


\begin{flushleft}
\textbf{4    CHAPTER ONE}
\end{flushleft}
research practice. This book also discusses some of the perils of conducting Net-based
research. While the Net provides many new opportunities to improve our research
practices, it also introduces new problems and challenges.
\begin{flushleft}
\textbf{WHAT DOES THE e IN e-RESEARCH MEAN?}
\end{flushleft}
We often joke that adding the letter ¢ in front of every noun we use is an unfortunate
distinction of the early years of this Internet technology era. We struggled with the
stigma of trendiness that will mark and date a text referring to e-research. In fact we fear
a visit from the “Society for the Preservation of the Other 25 Letters” when they see
the effusive use of the ¢ prefix used in this book! However, we think the term captures
some of the excitement, breadth, and diversity offered by an ever-increasing and some-
times bewildering set of new Net-based tools and techniques. Only a few years ago ¢
(as in email) meant a tool that was primarily text-based, operated on a relatively inse-
cure communications link, and provided a wide variation in performance and quality
of service. In education, ¢-applications focused on the lowest common denominators
so that students and faculty could access contents with even the slowest and most dated
of hardware. Convergence of audio, video, and multimedia channels to a Net-based
platform, which is continuing to fall in price and rise in power has resulted in an explo-
sion of applications in almost every domain, This has also resulted in a change of our
connotations of the Net or the e word. Generally, the e prefix means that the activity
or noun modified takes place on a high-speed, digital network that is available “any-
time/anywhere.” Today that network is the Internet.
\begin{flushleft}
\textbf{WHAT EDUCATIONAL RESEARCH ACTIVITIES DOES
e-RESEARCH ENCOMPASS?}
\end{flushleft}
The Net now supports a wide variety of communication modes and information pro-
cessing tools. As such, it is becoming easier to define the subset of behaviors that ean-
not be researched on the Net as opposed to those that can be the subject of research.
Not withstanding the dangers of missing novel ways of using the Net, we list below
some of the most obvious manifestations of e-research,

\begin{itemize}
\item
Distribution and retrieval of text-based surveys.l
\item
Open-ended or structured text-based interviews conducted via email or computer-
mediated conferencing
\item
Focus groups using real-time Net-based video or audio conferencing.
\item
Analysis of Web logs and other tracking tools for measurement and synthesis of
online activities.
\item
Net-based telephone interviews.
\item
Analysis of text transcripts of learning or social activities.
\item
Analysis of social behavior in virtual reality environments.
\item
Online assessment and/or evaluation of performance or knowledge.
\end{itemize}
\end{document}


