\documentclass [10pt,a4paper]{book}

\begin{document}

\begin{flushright}
\textbf{5  INTRODUCTION}
\end{flushright}
We expect that the increasing power and ubiquity of the Net coupled with its imagi-
native use by researchers will result in continuing expansions and variations of the
scope of research practiced around the globe.
\begin{flushleft}
\textbf{THE SPECIAL TASK OF e-RESEARCH}
\end{flushleft}
The networked world is awash in volumes of data. E-research helps us to convert this
data into information and present and disseminate this information in ways that allow
it to be transformed into knowledge and wisdom by the researchers, their sponsors,
educators, and the general public. The quantity of information produced, coupled
with the speed in which it can be accessed, filtered, sorted, and combined creates end-
less opportunity. However, this abundance forces ¢-researchers to be more selective
and critical of the veracity of the data they gather. In addition, it is becoming increas-
ingly apparent that we can no longer, if we ever could, gather all relevant data. Instead
we must make judicious decisions about which type and what quantity of data is most
helpful in answering our research questions.


E--research is more than a set of new research techniques. The quantum physicist
studying subatomic particles realizes that the very act of viewing these tiniest of parti-
cles disturbs and changes the objects, The e-researcher is a component of the Net. E-
researchers provide and create tools for analysis and conceptual understanding of
human behavior as it develops on the networks. In some cases the e-researcher is the
outside evaluator, in other contexts the practitioner e-researcher is both a participant
and researcher of the environment in which the research occurs, E-researchers are also
usually members of other Net communities, thas they bring their experience and
insights into the way online individuals and groups communicate and operate. They
act as Net-savvy artisans of a network culture. Informally, they interact with peers,
family, and coworkers—investing their time in the development of new skills and in the
process gaining “Net efficacy.”


E-research takes its place alongside e-commerce and ¢-learning as alternative
ways to act, understand, and create knowledge in a networked society. New tools
require new skills, but also allow creativity and an ability to manipulate the world in
different ways. These new tools span both the physical and temporal barriers. We are
accustomed to conceiving of technology spanning geography—after all, humans have
had nearly 150 years since the telegraph first allowed us to communicate in real time
over geographic distance. The Net easily meets this challenge. But equally, the Net
spans temporal distance. Users are now able to benefit from asynchronous interaction
through the tools of email and voicemail, or the capture and time shifting of audio or
visual presentation. New tools such as asynchronous voice conferencing and “video
capture” (an advanced form of picture mail) promise to allow full multimedia interac-
tion in asynchronous formats.


Asynchronous communication has also been with us for a long time. From St.
Paul’s letters to the early Christian church to the friendships that have grown and
flourished via pen pal letters—asynchronicity has provided a uniquely reflective means
by which humans communicate and by which we are communicating with you at this


\end{document}

